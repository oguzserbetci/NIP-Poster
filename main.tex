%%%%%%%%%%%%%%%%%%%%%%%%%%%%%%%%%%%%%%%%%
% Jacobs Landscape Poster
% LaTeX Template
% Version 1.0 (29/03/13)
%
% Created by:
% Computational Physics and Biophysics Group, Jacobs University
% https://teamwork.jacobs-university.de:8443/confluence/display/CoPandBiG/LaTeX+Poster
% 
% Further modified by:
% Nathaniel Johnston (nathaniel@njohnston.ca)https://v2.overleaf.com/6621156184mqqwsfbrcjkk
%
% This template has been downloaded from:
% http://www.LaTeXTemplates.com
%
% 
% Masaryk University presentation themes were downloaded from:
% https://www.overleaf.com/gallery/tagged/muni
%
% and ported into Jacobs Landscape Poster by:
% Jumaidil Awal (ideal1st.here@googlemail.com)
% 
% Jacobs Landscape Poster License:
% CC BY-NC-SA 3.0 (http://creativecommons.org/licenses/by-nc-sa/3.0/)
%
% Masaryk University's fibeamer theme license:
% Copyright 2015  Vít Novotný <witiko@mail.muni.cz>
% Faculty of Informatics, Masaryk University (Brno, Czech Republic)
% under Latex Project Public License
%
%%%%%%%%%%%%%%%%%%%%%%%%%%%%%%%%%%%%%%%%%

%----------------------------------------------------------------------------------------
%	PACKAGES AND OTHER DOCUMENT CONFIGURATIONS
%----------------------------------------------------------------------------------------

\documentclass[final]{beamer}

\usepackage[scale=0.8]{beamerposter} % Use the beamerposter package for laying out the poster



%\usetheme{confposter} % Use the confposter theme supplied with this template
\usetheme[faculty=chemo]{fibeamer} % Uncomment to use Masaryk University's fibeamer theme instead.

% \setbeamercolor{block title}{fg=nred,bg=white} % Colors of the block titles
% \setbeamercolor{block body}{fg=black,bg=white} % Colors of the body of blocks
%\setbeamercolor{block alerted title}{fg=white,bg=dblue!70} % Colors of the highlighted block titles
%\setbeamercolor{block alerted body}{fg=black,bg=dblue!10} % Colors of the body of highlighted blocks
% Many more colors are available for use in beamerthemeconfposter.sty

%-----------------------------------------------------------
% Define the column widths and overall poster size
% To set effective sepwid, onecolwid and twocolwid values, first choose how many columns you want and how much separation you want between columns
% In this template, the separation width chosen is 0.024 of the paper width and a 4-column layout
% onecolwid should therefore be (1-(# of columns+1)*sepwid)/# of columns e.g. (1-(4+1)*0.024)/4 = 0.22
% Set twocolwid to be (2*onecolwid)+sepwid = 0.464
% Set threecolwid to be (3*onecolwid)+2*sepwid = 0.708

\newlength{\sepwid}
\newlength{\onecolwid}
\newlength{\twocolwid}
\newlength{\threecolwid}
\setlength{\paperwidth}{46.8in} % A0 width: 46.8in
\setlength{\paperheight}{33.1in} % A0 height: 33.1in
\setlength{\sepwid}{0.01\paperwidth} % Separation width (white space) between columns
\setlength{\onecolwid}{0.23\paperwidth} % Width of one column
\setlength{\twocolwid}{0.451\paperwidth} % Width of two columns
\setlength{\threecolwid}{0.678\paperwidth} % Width of three columns
%\setlength{\topmargin}{-0.5in} % Reduce the top margin size
%-----------------------------------------------------------

\usepackage{graphicx}  % Required for including images

\usepackage{booktabs} % Top and bottom rules for tables

\usepackage {tikz}
\usetikzlibrary {positioning}
\usepackage{graphicx,caption}

\usepackage{subcaption}

\definecolor {processblue}{cmyk}{0.96,0,0,0}

%----------------------------------------------------------------------------------------
%	TITLE SECTION 
%----------------------------------------------------------------------------------------

\title{Neural mechanisms of risk-sensitive choice and reinforcement
learning under uncertainty} % Poster title

\author{Jan Botsch, Emil Azadian, Oǧuz Şerbetci, Vlad Frasineanu, Stephan Tietz – Neural Information Project, Neural Information Processing Group, TU Berlin} % Author(s)

%----------------------------------------------------------------------------------------

\begin{document}

\addtobeamertemplate{block end}{}{\vspace*{2ex}} % White space under blocks
\addtobeamertemplate{block example end}{}{\vspace*{2ex}} % White space under example blocks
\addtobeamertemplate{block alerted end}{}{\vspace*{2ex}} % White space under highlighted (alert) blocks

\setlength{\belowcaptionskip}{2ex} % White space under figures
\setlength\belowdisplayshortskip{2ex} % White space under equations
%\begin{darkframes} % Uncomment for dark theme, don't forget to \end{darkframes}

\begin{frame} % The whole poster is enclosed in one beamer frame

%==========================Begin Head===============================
  \begin{columns}
   \begin{column}{\linewidth}
    \vskip1cm
    \centering
    \usebeamercolor{title in headline}{\color{fg}\Huge{\textbf{\inserttitle}}\\[0.5ex]}
    \usebeamercolor{author in headline}{\color{fg}\Large{\insertauthor}\\[1ex]}
    \usebeamercolor{institute in headline}{\color{fg}\large{\insertinstitute}\\[1ex]}
    \vskip1cm
   \end{column}
   \vspace{1cm}
  \end{columns}
 \vspace{1cm}

%==========================End Head===============================

\begin{columns}[t] % The whole poster consists of three major columns, the second of which is split into two columns twice - the [t] option aligns each column's content to the top

\begin{column}{\sepwid}\end{column} % Empty spacer column

\begin{column}{\onecolwid} % The first column

%----------------------------------------------------------------------------------------
%	Motivation
%----------------------------------------------------------------------------------------

\begin{exampleblock}{Motivation}
Researching human risk behavior under uncertainty is important to build algorithms that are capable of operating in high risk environments.



\end{exampleblock}

%----------------------------------------------------------------------------------------
%	INTRODUCTION
%----------------------------------------------------------------------------------------

\begin{exampleblock}{Background}
%TODO Put Risk and Utility in smaller font
%TODO Change text about utility and risk sensitivites to a comic
%TODO Change text color of captions
%TODO Include information about exp UF and include the formula


\Large{Risk and Utility}

\normalsize

\begin{itemize}
    \item Risk is induced by choice with uncertain outcome.
    \item People behave \textbf{risk seeking}, \textbf{risk neutral} or \textbf{risk averse}.
    % \item Risk can exist without the danger of loss (e.g. loosing money).
\end{itemize}

\begin{itemize}
    \item People map money to utility by \textbf{utility functions}.
    %\item The same amount of money can have different utility for different people.
    \item Curvature of function defines risk profile.
\end{itemize}


\begin{figure}
  \centering
    \includegraphics[width=0.9\textwidth]{img/background/riskaversion.jpg}
  \caption{Choice from a lottery where you gain 10\$ or 50\$ by 50\% or a gift of 30\$ for sure. The utility for the fixed reward (point A) is higher than the expected utility of the lottery (point B).}
  
\end{figure}


%TODO Change this block to a plot.
\Large{(Partially Observable) Markov Decision Process}

\normalsize

\begin{itemize}
    \item Making decisions in a partially observable environment.
    \item Work on probability distribution over states rather than actual state.
\end{itemize}
\begin{figure}
  \centering
    \includegraphics[width=0.9\textwidth]{img/background/POMDP}
  \caption{Cartoon of a partially observable environment. Instead of states the agent only gets observations as inputs.}
  
\end{figure}


%\begin{itemize}
%    \item Set of States $\mathcal{S}$ (Terminal and Non Terminal)
%    \item Set of Actions $\mathcal{A}$
%    \item Probabilistic Transitions depending on tuple $\mathcal{S} x \mathcal{A}$
%    \item Reward function $R(s,a)$
%\end{itemize}

%For partially observability states are hidden but produce observations:
%\begin{itemize}
%    \item Observation space $\mathcal{O}$
%    \item Observation function $p(o | s, a)$
%\end{itemize}

% add reference: http://www.cassandra.org/arc/papers/aaai94.pdf


\end{exampleblock}



%----------------------------------------------------------------------------------------
%	REFERENCES
%----------------------------------------------------------------------------------------

\begin{exampleblock}{References}

\nocite{*} % Insert publications even if they are not cited in the poster
\small{\bibliographystyle{unsrt}
\bibliography{sample}\vspace{1cm}}
\end{exampleblock}

%----------------------------------------------------------------------------------------
%	ACKNOWLEDGEMENTS
%----------------------------------------------------------------------------------------

%\setbeamercolor{block title}{fg=red,bg=white} % Change the block title color

%\begin{exampleblock}{Acknowledgements}

%\small{\rmfamily{Nam mollis tristique neque eu luctus. Suspendisse rutrum congue nisi sed convallis. Aenean id neque dolor. Pellentesque habitant morbi tristique senectus et netus et malesuada fames ac turpis egestas.}} \\

%\end{exampleblock}

\begin{block}{Acknowledgements}

This work was supported by [Department Name]. The authors gratefully acknowledge the helpful discussions and technical assistance provided by [Rong], [Vaios].
We also thank the WZB department for assisting us during the execution of the experiment.

\end{block}

%----------------------------------------------------------------------------------------

\end{column} % End of the first column

\begin{column}{\sepwid}\end{column} % Empty spacer column

\begin{column}{\onecolwid} % The first column

%----------------------------------------------------------------------------------------
%	OBJECTIVES
%----------------------------------------------------------------------------------------

\begin{exampleblock}{Objectives}

\begin{itemize}
\item \textbf{Examine} behavior of people under risky choice.
\item \textbf{Reassess} the assumption that exponential utility function fully explains human risk behavior.
\item \textbf{Create} an artificial agent that behaves in risk-sensitive ways similar to humans, using various utility functions.
\end{itemize}

\end{exampleblock}

%----------------------------------------------------------------------------------------
%	METHODS
%----------------------------------------------------------------------------------------

\begin{exampleblock}{Methods}



\Large{The Experiment}
%TODO Get rid of 4panel, put the top left picture, include text/caption explaining the setting based on picture added
\normalsize
\begin{figure}
\begin {center}
\begin {tikzpicture}[-latex ,auto ,node distance =3cm and 4cm ,on grid ,
semithick ,
state/.style ={ circle ,fill=black!20, minimum width =3 cm}]
\node[state] (C){$Sold$};
\node[state] (A) [above left=of C,align=center] {Reces\\sion};
\node[state] (B) [above right =of C,align=center] {Boom\\ing};
\coordinate[below of=A] (AA);
\coordinate[below of=B] (BB);
\coordinate[below of=AA] (D);
\coordinate[below of=BB] (E);


\path (A) edge [loop left, line width=2mm, align=center] node[left] {wait \\ $0.86$} (A);
\path (A) edge [bend left = -25,line width=2mm,align=center] node[below =0.25 cm] {sell\\$1.0$} (C);
\path (A) edge [bend left =25,line width=2mm,align=center] node[above] {wait\\$0.14$} (B);

\path (B) edge [loop right,line width=2mm,align=center] node[right] {wait\\$1.0$} (B);
\path (B) edge [bend right = -25,line width=2mm,align=center] node[below =0.25 cm] {sell\\$1.0$} (C);

%\fill[gray!40!white, opacity=0.5] (-6,-1) rectangle (5,6);

\path (A) edge [bend right =25,line width=2mm, dashed] node[left] {$Observation$} (D);
\path (B) edge [bend left  =25,line width=2mm, dashed] node[right] {$Observation$} (E);
\end{tikzpicture}
\end{center}
\end{figure}

\begin{itemize}
    \item In a simulated housing market, participants needed to decide whether to sell or to wait.
    \item Waiting decreases risk, but increases costs.
    \item 24 participants performed three different scenarios.
\end{itemize}

\begin{figure}
  \centering
    \includegraphics[width=0.9\textwidth]{img/methods/experiment_obs_1.png}
  \caption{Part of the UI used in the experiment showing the observation history.}
\end{figure}


\Large{The Agent}
\normalsize
\begin{itemize}
    \item Simplify POMDP into a MDP by Bayesian estimates.
    \begin{figure}
        \centering
        \includegraphics{img/belief_table.pdf}
        \caption{Caption}
        \label{fig:my_label}
    \end{figure}
    \item Solve the MDP with value iteration on an augmented state space.
    % TODO formula for value iteration, expected values separated if it doesn't fit
    \[V^{(n-1)}(b,w) =\ldots\]
    \[Q^{(n)}_{U} (b, w) = \max(\int_{-\inf}^{\inf}{P(o|b)}) \]
    \item Different utility functions.
    % formulas for different utility functions.
\end{itemize}


\end{exampleblock}

\end{column} % End of column 2

%----------------------------------------------------------------------------------------
%	IMPORTANT RESULT
%----------------------------------------------------------------------------------------

%\begin{alertblock}{Important Result}

%Lorem ipsum dolor \textbf{sit amet}, consectetur adipiscing elit. Sed commodo molestie porta. Sed ultrices scelerisque sapien ac commodo. Donec ut volutpat elit.

%\end{alertblock} 

%----------------------------------------------------------------------------------------

\begin{column}{\sepwid}\end{column} % Empty spacer column

\begin{column}{\onecolwid} % The first column within column 2 (column 2.1)

\begin{exampleblock}{Results}

%----------------------------------------------------------------------------------------
%	RESULTS
%----------------------------------------------------------------------------------------

\begin{exampleblock}{Results}

\begin{figure}
\includegraphics[width=0.45\linewidth]{img/avg_reward.png}
\includegraphics[width=0.45\linewidth]{img/avg_waiting.png}
\caption{Average reward and waiting times in different experiment scenarios.}
\end{figure}

\begin{figure}
    \includegraphics[width=0.8\linewidth]{img/rational_policy.png}
\caption{Value function and policy of a traditional RL agent with risk-neutral behaviour in the expensive observation scenario. The agent sells at a fixed belief regardless of waiting time.}
\end{figure}

\begin{figure}
    \includegraphics[width=0.8\linewidth]{img/sine_policy.png}
    \caption{Value function and policy of a RL agent with hyperbolic sine utility function in the expensive observation scenario. The agent sells at a fixed time step regardless of its belief.}
\end{figure}

\begin{figure}
    \includegraphics[width=0.8\linewidth]{img/dynamic_policy.png}
    \caption{Value function and policy of a RL agent with hyperbolic sine utility function in the expensive observation scenario. The agent sells depending on both time and its belief.}
\end{figure}


%\begin{table}
%\vspace{2ex}
%\begin{tabular}{l l l}
%\toprule
%\textbf{Treatments} & \textbf{Response 1} & \textbf{Response 2}\\
%\midrule
%Treatment 1 & 0.0003262 & 0.562 \\
%Treatment 2 & 0.0015681 & 0.910 \\
%Treatment 3 & 0.0009271 & 0.296 \\
%\bottomrule
%\end{tabular}
%\caption{Table caption}
%\end{table}

\end{exampleblock}

%----------------------------------------------------------------------------------------

\end{column} % End of the third column

\begin{column}{\sepwid}\end{column} % Empty spacer column

\begin{column}{\onecolwid} % The third column

%----------------------------------------------------------------------------------------
%	DISCUSSION
%----------------------------------------------------------------------------------------

\begin{figure}
    \includegraphics[width=0.8\linewidth]{img/fit.png}
\caption{Examples from three different behaviors observed and replicated with RL agents.}
\end{figure}



% \begin{exampleblock}{Discussion}

% \begin{itemize}
%     \item
% \end{itemize}

% \end{exampleblock}

%----------------------------------------------------------------------------------------
%	ADDITIONAL INFORMATION
%----------------------------------------------------------------------------------------

\begin{exampleblock}{Conclusion}

\begin{itemize}
    \item Participants were categorized into three different groups of risk-behavior.
    \item Exponential utility cannot model all human risk-behavior.
    % TODO fill up
    \item Agents acting with different risk-profile perform ... 
\end{itemize}

\end{exampleblock}

\begin{exampleblock}{Future Work}

\begin{itemize}
    \item End-to-end method for deriving utility functions from human behaviour (i.e. solve the inverse problem).
    \item Demonstration in high-risk applications such as flight control.
\end{itemize}

\end{exampleblock}

\end{column} % End of the third column

\begin{column}{\sepwid}\end{column} % Empty spacer column

\end{columns} % End of all the columns in the poster

\end{frame} % End of the enclosing frame
%\end{darkframes} % Uncomment for dark theme
\end{document}
